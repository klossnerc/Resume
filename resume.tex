% LaTeX code for rendering the resume is distributed under the MIT license.
% See LICENSE.txt. It means you can use the code for whatever you want,
% including your own resume, but I'm not liable if it catches your computer on
% fire.

% Template originally developed by Emily Dunham
% https://github.com/edunham/resume/blob/master/resume.tex

\documentclass[11pt]{article}
\topmargin=0.0in
\textwidth=6.5in

\topmargin 0pt
\advance \topmargin by -\headheight
\advance \topmargin by -\headsep

\setlength{\hoffset}{-0.25in}

\oddsidemargin 0pt
\evensidemargin \oddsidemargin
\marginparwidth 0.5in


\headheight=0pt
\headsep=0pt
\textheight=10.5in

\pagenumbering{gobble}
\setlength{\parindent}{0pt}

\usepackage[normalem]{ulem} % for the underlines
\usepackage[compact]{titlesec}
        
\newcommand{\heading}[1]{
    \section*{\uline{\hfill #1}}
}

\newcommand{\squish}{
    \setlength{\itemsep}{0.5pt}
    % tweak parskip value to adjust size on page
    \setlength{\parskip}{0pt}
    \setlength{\parsep}{0.5pt}
}

% naming it 'date' would conflict with builtins
\newcommand{\when}[1]{
    \hfill \texttt{#1}
}

% place, optional title, date
\newcommand{\experience}[4]{
    \ifx&#2&
        \item[{#1}]
        \when{#4}
    \else
        \ifx&#3&
            \item[{#1}, \emph{#2}]
            \when{#4}
        \else
            \item[{#1}, \emph{#2}]
            \when{#4}
            \item{#3}
            \when{}
        \fi
    \fi
}

\newcommand{\contact}[4]{
    \centerline{
        \large       
        \texttt{#1}
        $\bullet$
        \texttt{#2}
        $\bullet$
        \texttt{#3}
    }
    \centerline{
        \emph{#4}
    }
}

\newcommand{\skill}[2]{
    \textbf{#1} \hfill #2
}

% http://www.parashift.com/c++-faq-lite/latex-macros.html
\newcommand{\CPP}{
    C\hspace{-.05em}\raisebox{.4ex}{\tiny\bf +}\hspace{-.10em}\raisebox{.4ex}{\tiny\bf +}
}

\begin{document}

\centerline{{\Huge \bf Catrina Klossner}}
\bigskip

\contact{catrina@klossner.org}
        {github.com/klossnerc}
        {(503) 421-6704}
        {Portland, Oregon}

%\heading{Skills}%%%%%%%%%%%%%%%%%%%%%%%%%%%%%%%%%%%%%%%%%%%%%%%%%%%%%%%%%%%%%%
%
%\skill{Development}{Python, C/\CPP, HTML/CSS/Javascript, Django, Flask}
%
%\skill{Collaboration}{IRC, Git \& GitHub, Documentation (.mkd, .rst, Sphinx),
%Redmine/Jira/RT} 
%
%\skill{Robotics}{Mechanical design \& prototyping, 3D modeling, OpenCV, Arduino}
%
%\skill{Leadership}{Teaching, public speaking, event planning, curriculum
%design}

\heading{Education}%%%%%%%%%%%%%%%%%%%%%%%%%%%%%%%%%%%%%%%%%%%%%%%%%%%%%%%%%%

\begin{description}
\squish   
\experience{Oregon State University}
           {BS Computer Science}
           {GPA: 3.86}
           {June 2015}

	- Implemented in a modified Linux kernel an elevator io-scheduler and a best-fit algorithm for the Simple List of Blocks (SLOB) 

	- Designed and implemented in C a translator which compiled a Lisp-like language to gForth 

	- Demonstrated basic proficiency with shaders, perspectives, lighting, and objects in OpenGL 


\experience{Oregon State University}
           {BS Electrical \& Computer Engineering}
           {GPA: 3.86}
           {June 2015}

	- Led a team of three in the design and development of a wearable pendant from intitial concept to prototyping 

	- Built and programmed in C a FM alarm clock-radio on an Atmega128 that communicated with an Atmega48 and a Texas Instruments LM73 temperature sensor.

	- Calculated the benefits and cost of current computer architecture design techniques such as pipelining 

\end{description}

\skill{Class List}{Operating Systems 1\&2, Translators, Computer Architecture, Microprocessor System Design, Software Engineering 1\&2, Programming Language Fundamentals, Computer Organization \& Assembly Language}


\heading{Employment}%%%%%%%%%%%%%%%%%%%%%%%%%%%%%%%%%%%%%%%%%%%%%%%%%%%%%%%%%%

\begin{description}
\squish
\experience{Air-Weigh}
           {Software Engineer Intern}
           {}
           {June 2014 - December 2014}
 
	- Pitched the design for a new communication protocol to high-level managers

	- Developed a physical-layer protocol bridgre from initial requirements to final software and hardware testing

	- Documented a new tool for initial software set-up
  
\experience{Garmin AT}
           {Software Engineer Intern}
           {}
           {April 2013 - September 2013}
 
	- Implemented a search feature for a legacy database to facilitate access to 21 years of data

	- Identified error cases and implemented autonomous unit tests to improve code coverage by 80 percent

	- Lead development of a serial-peripheral interface (SPI) driver for multiple projects

	- Taught overview of key kernel development concepts to software developers
 
\end{description}


\heading{Skills}%%%%%%%%%%%%%%%%%%%%%%%%%%%%%%%%%%%%%%%%%%%%%%%%%%%%%%%%%%%%%%

\skill{Languages}{Fluent in C, familiar with Java, played with Haskell and Python}

\skill{Tools}{Experienced with oscilloscope, played with GNU debugger}


\heading{Extracurriculars}%%%%%%%%%%%%%%%%%%%%%%%%%%%%%%%%%%%%%%%%%%%%%%%%%%%%%

\begin{description}
\squish
\experience{Phi Sigma Rho - Engineering Sororoity}
           {}
           {}
           {2010 - 2015}

	- Mentored and tutored younger Electrical Engineering and Computer Science students

	- Maintained the sorority's website for 2 years as co-webmaster


\experience{FIRST Tech Challenge}
           {Volunteer}
           {}
           {2012 - 2015}

	- Enforced safety protocols and handled paperwork as pit administrator

	- Helped keep event on schedule by wrangling teams as team queuer


\experience{Oregon State University Robotics Club}
           {}
           {}
           {2010 - 2012}

	- Assembled 2 different robots using the Oregon State Robotic Kits

	- Maintained the club website for 1 year as webmaster


\end{description}
\end{document}
